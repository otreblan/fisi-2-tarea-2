\documentclass[12pt, twoside]{article}
% Opciones {{{
\usepackage[pdfa, pdfusetitle, unicode=true]{hyperref}
\usepackage[spanish]{babel}
\usepackage[margin=2.5cm, a4paper]{geometry}
\usepackage{luacode}
\usepackage[shortlabels]{enumitem}
\usepackage{import}
\usepackage{xcolor}
\usepackage{fontspec}
\usepackage[mark, raisemark=0.02\paperheight, marknotags]{gitinfo2}
\usepackage{setspace}

\doublespacing

% Btw I use arch
\setmonofont{InconsolataGo Nerd Font}

\newcommand{\btw}{{\color{arch}\texttt{}} }
\newcommand{\git}{{\color{git}\texttt{}} }

\renewcommand{\gitMarkPref}{{\Large\git git}}

% Esto sirve para poner ecuaciones
\usepackage{mathtools}

% Esto sirve para poner imágenes{{{
\usepackage{graphicx}
\usepackage{svg}
\usepackage{subcaption}

\usepackage{float}
\usepackage{pgfplots}

\pgfplotsset{compat=1.16}
\graphicspath{ {ima/} }
%}}}
% Colores de los links {{{
\definecolor{red}{HTML}{F22C40}
\definecolor{green}{HTML}{5AB738}
\definecolor{yellow}{HTML}{D5911A}
\definecolor{blue}{HTML}{407EE7}
\definecolor{magenta}{HTML}{6666EA}
\definecolor{cyan}{HTML}{00AD9C}
\definecolor{arch}{HTML}{1793D1}
\definecolor{git}{HTML}{F54D27}

\hypersetup{
	colorlinks=true,
	linkcolor=blue,
	urlcolor=cyan,
	citecolor=magenta,
}
%}}}
% Esto controla a la cabecera {{{
\usepackage{fancyhdr}

\pagestyle{fancy}
\fancyhf{}
\renewcommand{\headrulewidth}{0pt}
\chead{ \textbf{\normalsize{Física II} }}
%\fancyhf[HL]{\includesvg[height=0.8\headheight]{Utec.svg}}
\fancyhf[FL]{\textbf{\thepage}}
\setlength{\headheight}{20pt}
\setlength{\textheight}{675pt}
%}}}
% Título {{{
\title{\textbf{<++>}}
% Aqui hay que poner a los autores
\author{
		Alberto Oporto Ames\\
		\texttt{alberto.oporto@utec.edu.pe}\\
		%\and <++>\\
		%\texttt{<++>}\\
		}
%}}}
%}}}
% Aquí empieza el documento{{{
\begin{document}
%\maketitle
\textbf{Alberto Oporto Ames \#139}
\thispagestyle{fancy}

% Preguntas {{{
\section{Preguntas}%
\label{sec:Preguntas}
%\setcounter{subsection}{2}

\subsection*{¿Por qué los camiones que transportan combustible arrastran una cadena?}%

\subsection*{Una persona está manejando sobre un terreno húmedo cuando de repente
empieza una tormenta eléctrica
¿Es seguro permanecer dentro del auto? ¿Por qué?}%

\subsection*{¿Cómo serán las líneas de campo y equipotenciales generadas por una
varilla cargada uniformemente? ¿Y para una carga triangula?
Graficalas}%


% }}}
% Problemas {{{
\section{Problemas}%
\label{sec:Problemas}
\subsection*{El siguente arco de circunferencia, de díamentro $d = 1m$ y centro $O$,
se encuentra cargado positivamente con una densidad lineal $\lambda = 20\mu \frac{C}{m}$.
Determina:}%
\begin{enumerate}[label=\textbf{\alph*)}]
	\item El campo eleéctrico en el punto $O$.
	\item La fuerza eléctrica sobre una carga puntual $q(-2\mu C)$ que se sitúa en el punto $O$.
	\item El potencial eléctrico en el punto $O$.
\end{enumerate}

\subsection*{Una esfera pequeña de metal tiene una carga neta de $q_1 =-3.5\mu C$
y se mantiene en posición estacionaria por medio de soportes aislados.
Una segunda esfera metálica también pequeña con carga neta de $q_2 =-0.2mC$ y masa de $150g$ es
lanzada horizontalmente hacia $q_1$ desde una distancia de $80cm$ con una rapidez de
$22\frac{m}{s}$. Determina lo siguiente:}%
\begin{enumerate}[label=\textbf{\alph*)}]
	\item Las fuerzas que actúan sobre $q_2$.
	\setcounter{enumi}{2}
	\item La distancia mínima a la que llegan a estar las esferas.
\end{enumerate}

\subsection*{La esfera hueca mostrada en la figura es de material no conductor,
radio externo $2a$, y posee densidad volumétrica de carga constante.
La cavidad también es esférica, de radio $a$, con centro a una distancia
$a$ del centro. Dtermina:}%
\begin{enumerate}[label=\textbf{\alph*)}]
	\item El campo eléctrico en el punto $(0;4a)$
	\item El potencial eléctrico en el punto $(4a;0)$
\end{enumerate}

% }}}
\vfill
Código fuente: \url{https://github.com/otreblan/fisi-2-tarea-2}

\end{document}
%}}}
