\documentclass[10pt, twoside]{article}
% Opciones {{{
\usepackage[pdfa, pdfusetitle, unicode=true]{hyperref}
\usepackage[spanish]{babel}
\usepackage[margin=2.5cm, a4paper]{geometry}
\usepackage{luacode}
\usepackage[shortlabels]{enumitem}
\usepackage{import}
\usepackage{xcolor}
\usepackage{fontspec}
\usepackage[mark, raisemark=0.02\paperheight, marknotags]{gitinfo2}
\usepackage{setspace}
\usepackage{floatrow}

\doublespacing

% Btw I use arch
\setmonofont{InconsolataGo Nerd Font}

\newcommand{\btw}{{\color{arch}\texttt{}} }
\newcommand{\git}{{\color{git}\texttt{}} }

\renewcommand{\gitMarkPref}{{\Large\git git}}

% Esto sirve para poner ecuaciones
\usepackage{mathtools}
\usepackage{cancel}
\usepackage{tikz}
\usetikzlibrary{spy, backgrounds}
\allowdisplaybreaks

% Esto sirve para poner imágenes{{{
\usepackage{graphicx}
\usepackage{svg}
\usepackage{subcaption}

\usepackage{float}
\usepackage{pgfplots}

\pgfplotsset{compat=1.16}
\graphicspath{ {ima/} }
%}}}
% Colores de los links {{{
\definecolor{red}{HTML}{F22C40}
\definecolor{green}{HTML}{5AB738}
\definecolor{yellow}{HTML}{D5911A}
\definecolor{blue}{HTML}{407EE7}
\definecolor{magenta}{HTML}{6666EA}
\definecolor{cyan}{HTML}{00AD9C}
\definecolor{arch}{HTML}{1793D1}
\definecolor{git}{HTML}{F54D27}

\hypersetup{
	colorlinks=true,
	linkcolor=blue,
	urlcolor=cyan,
	citecolor=magenta,
}
%}}}
% Esto controla a la cabecera {{{
\usepackage{fancyhdr}

\pagestyle{fancy}
\fancyhf{}
\renewcommand{\headrulewidth}{0pt}
\chead{ \textbf{\normalsize{Física II} }}
%\fancyhf[HL]{\includesvg[height=0.8\headheight]{Utec.svg}}
\fancyhf[FL]{\textbf{\thepage}}
\setlength{\headheight}{20pt}
\setlength{\textheight}{675pt}
%}}}
% Título {{{
\title{\textbf{<++>}}
% Aqui hay que poner a los autores
\author{
		Alberto Oporto Ames\\
		\texttt{alberto.oporto@utec.edu.pe}\\
		%\and <++>\\
		%\texttt{<++>}\\
		}
%}}}
%}}}
% Aquí empieza el documento{{{
\begin{document}
%\maketitle
\textbf{Alberto Oporto Ames \#139}
\thispagestyle{fancy}

% Preguntas {{{
\section{Preguntas}%
\label{sec:Preguntas}
%\setcounter{subsection}{2}

\subsection*{¿Por qué los camiones que transportan combustible arrastran una cadena?}%
Para neutralizar la carga que obtienen con el roce del viento.

\subsection*{Una persona está manejando sobre un terreno húmedo cuando de repente
empieza una tormenta eléctrica
¿Es seguro permanecer dentro del auto? ¿Por qué?}%
Sí.
Porque las llantas sirven como aislante.

\subsection*{¿Cómo serán las líneas de campo y equipotenciales generadas por una
varilla cargada uniformemente? ¿Y para una carga triangular?
Graficalas}%

\begin{figure}[H]
	\centering
	\begin{floatrow}
		\ffigbox{\includesvg[width=\linewidth]{dibujo}}{}
		\ffigbox{\includesvg[width=0.8\linewidth]{dibujo2}}{}
	\end{floatrow}
\end{figure}


% }}}
% Problemas {{{
\section{Problemas}%
\label{sec:Problemas}
% Problema 3{{{
\subsection*{El siguente arco de circunferencia, de díamentro $d = 1m$ y centro $O$,
se encuentra cargado positivamente con una densidad lineal $\lambda = 20\mu \frac{C}{m}$.
Determina:}%
\begin{enumerate}[label=\textbf{\alph*)}]
	\item El campo eléctrico en el punto $O$.
		\begin{align*}
			dE &= k \frac{dQ}{r^2} \\
			%
			dE &= k \frac{\lambda \cancel{r}d\theta}{r^{\cancel{2}} } \\
			%
			d\vec{E} &= k \frac{\lambda}{r} d\theta*(\cos{\theta}(-\hat{\imath})+\sen{\theta}(-\hat{\jmath}))\\
			%
			\int^{\vec{E}}_{0}d\vec{E} &= k \frac{\lambda}{r}
			\int^{\frac{150°\pi}{180°}}_{\frac{30°\pi}{180°}}d\theta*(\cos{\theta}(-\hat{\imath})+\sen{\theta}(-\hat{\jmath}))\\
			%
			\vec{E} &= k \frac{\lambda}{r} *\Big{|}^{\frac{150°\pi}{180°}}_{\frac{30°\pi}{180°}}
			\Big{[}\sen{\theta}(-\hat{\imath})-\cos{\theta}(-\hat{\jmath})\Big{]}\\
			%
			\vec{E} &= 9*10^9 \frac{N\cancel{m^2}}{C^{\cancel{2}} }*
			20*10^{-6} \frac{2\cancel{C}}{\cancel{m^2}}
			\Big{(}
			\cancel{(\sen{\frac{150°\pi}{180°}}-\sen{\frac{30°\pi}{180°}})(-\hat{\imath})}-
			(\cos{\frac{150°\pi}{180°}}-\cos{\frac{30°\pi}{180°}})(-\hat{\jmath})
			\Big{)}\\
			%
			\Aboxed
			{
				\vec{E} &= -\Big{(}62.35*10^4 \frac{N}{C}\Big{)}\hat{\jmath}
			}
		\end{align*}
	\item La fuerza eléctrica sobre una carga puntual $q(-2\mu C)$ que se sitúa en el punto $O$.
		\begin{align*}
			\vec{F} &= \vec{E}*q\\
			\vec{F} &= -\Big{(}62.35*10^4 \frac{N}{\cancel{C}}\Big{)}\hat{\jmath}*-2*10^{-6}\cancel{C}\\
			\Aboxed
			{
				\vec{F} &= (124.70*10^{-2}N)\hat{\jmath}
			}
		\end{align*}
	\item El potencial eléctrico en el punto $O$.
		\begin{align*}
			dV &= k \frac{dQ}{r} \\
			dV &= k \frac{\lambda*(d\theta*\cancel{r})}{\cancel{r}}\\
			%
			\int^V_0dV &= k \lambda \int^{\frac{150°\pi}{180°}}_{\frac{30°\pi}{180°}}d\theta\\
			%
			V &= 9*10^9 \frac{Nm^{\cancel{2}} }{C^{\cancel{2}} }*20*10^{-6} \cancel{\frac{C}{m}}*\Big{(}
			\frac{5\pi}{6} - \frac{\pi}{6}
			\Big{)}\\
			%
			V &= 18*10^4* \frac{4\pi}{6}* \frac{Nm}{C} \\
			\Aboxed
			{
				V &= 12\pi*10^4* \frac{Nm}{C}
			}
		\end{align*}
\end{enumerate}
% }}}
% Problema 4{{{
\subsection*{Una esfera pequeña de metal tiene una carga neta de $q_1 =-3.5\mu C$
y se mantiene en posición estacionaria por medio de soportes aislados.
Una segunda esfera metálica también pequeña con carga neta de $q_2 =-0.2mC$ y masa de $150g$ es
lanzada horizontalmente hacia $q_1$ desde una distancia de $80cm$ con una rapidez de
$22\frac{m}{s}$. Determina lo siguiente:}%
\begin{enumerate}[label=\textbf{\alph*)}]
	\item Las fuerzas que actúan sobre $q_2$.
		\begin{align*}
			\vec{P} &= -1.5N\hat{\jmath} &&\text{Peso}\\
			\vec{F} &= k \frac{q_1q_2}{r(t)^2}*\hat{d}(t)_{q_1q_2} &&\text{Fuerza de $q_1$ a $q_2$}
		\end{align*}
	\setcounter{enumi}{2}
	\item La distancia mínima a la que llegan a estar las esferas.
		\subitem Código fuente del simulador que programé:
			\url{https://github.com/otreblan/fisi-2-tarea-2/tree/master/simulador}

			% Gráfico
			\shorthandoff{"}
			% Líneas {{{
			\begin{figure}[H]
				\centering
				\begin{tikzpicture}
					[
					spy using outlines={rectangle,
						magnification=5,
						width=5cm,
						height=2.5cm,
						connect spies
					}
					]
					\begin{axis}
						[
						width=0.9\linewidth,
						height=6cm,
						axis y line=left,
						axis x line=bottom,
						axis line style = ultra thick,
						ymin=-15,
						xlabel={Tiempo desde el inicio $(s)$},
						ylabel={Distancia entre $q_1$ y $q_2$ $(m)$},
						ymajorgrids=true]
						\addplot[
							restrict x to domain=0:0.5,
							color=red,
							mark=*,
							mark options={scale=0.4},
							%smooth,
							line width=0.1pt
							]
							table[
							col sep=comma,
							y expr=\thisrowno{3}*1.0,
							x index=0
							]{simulacion.csv};
						\addplot[
							restrict x to domain=0.040:0.045,
							color=black,
							mark=*,
							mark options={scale=0.4},
							nodes near coords={\fontsize{5pt}{0pt}\selectfont
							$\pgfmathprintnumber{\pgfkeysvalueof{/data point/y}}m$},
							nodes near coords align={below}
							]
							table[
							col sep=comma,
							y expr=\thisrowno{3}*1.0,
							x index=0
							]{simulacion.csv};
						\coordinate (ripple) at (0.04,-0.5);
						\coordinate (lupa) at (0.35,-4);
						\end{axis}
						\spy[black] on (ripple) in node[fill=white] at (lupa);
				\end{tikzpicture}
			\end{figure}
			%}}}
			\shorthandon{"}
\end{enumerate}
% }}}
% Problema 7 {{{
\subsection*{La esfera hueca mostrada en la figura es de material no conductor,
radio externo $2a$, y posee densidad volumétrica de carga constante.
La cavidad también es esférica, de radio $a$, con centro a una distancia
$a$ del centro. Determina:}%
\begin{align*}
	\rho_e & && \text{Densidad volumétrica}\\
	V_{on} & && \text{Volumen de la esfera}
\end{align*}
\begin{enumerate}[label=\textbf{\alph*)}]
	\item El campo eléctrico en el punto $(0;4a)$
		\begin{align*}
			\sum\vec{E} &= \vec{E}_1 + \vec{E}_2\\
			\\
			\vec{E}_1 &= k \frac{\rho_eV_{o1}}{d_1^2}\hat{u}\\
			\vec{E}_2 &= k \frac{-\rho_eV_{o2}}{d_2^2}\hat{u}\\
			\\
			\sum\vec{E} &= k\rho_e\Big{(}\frac{V_{o1}}{d_1^2}-\frac{V_{o2}}{d_2^2}\Big{)}\hat{u}\\
			\sum\vec{E} &= k\rho_e\Big{(}\frac{4\pi 8a^3}{3*16a^2}-\frac{4\pi a^3}{3*9a^2}\Big{)}\hat{u}\\
			\sum\vec{E} &= k\rho_e\Big{(}\frac{2\pi a}{3}-\frac{4\pi a}{27}\Big{)}\hat{u}\\
			\sum\vec{E} &= k\rho_e\Big{(}\frac{14\pi a}{27}\Big{)}\hat{u}\\
			\sum\vec{E} &= 9*10^9\rho_e\Big{(}\frac{14\pi a}{27}\Big{)}\hat{\jmath}\\
			\Aboxed
			{
				\sum\vec{E} &= \rho_e\Big{(}\frac{14\pi a}{3}\Big{)}10^9\hat{\jmath}
			}
		\end{align*}
	\item El potencial eléctrico en el punto $(4a;0)$
		\begin{align*}
			\sum V &= V_1+V_2\\
			\\
			V_1 &= k \frac{\rho_eV_{o1}}{d_1}\\
			V_2 &= k \frac{-\rho_eV_{o2}}{d_2}\\
			\\
			\sum V &= k\rho_e \Big{(} \frac{V_{o1}}{d_1}- \frac{V_{o2}}{d_2} \Big{)}\\
			\sum V &= k\rho_e \Big{(} \frac{\cancel{4}\pi 8a^3}{3*\cancel{4}a}- \frac{4\pi a^3}{ \sqrt{17}a} \Big{)}\\
			\sum V &= k\rho_e \Big{(} \frac{\pi 8a^2}{3}- \frac{4\pi a^2}{ \sqrt{17}} \Big{)}\\
			\sum V &= k\rho_e \Big{(} \frac{4\pi a^2(2\sqrt{17}-3)}{3\sqrt{17}} \Big{)}\\
			\Aboxed
			{
				\sum V &= \rho_e \Big{(} \frac{12\pi a^2(2\sqrt{17}-3)}{\sqrt{17}} \Big{)}10^9
			}
		\end{align*}
\end{enumerate}
% }}}
% }}}
\vfill
Código fuente: \url{https://github.com/otreblan/fisi-2-tarea-2}

\end{document}
%}}}
